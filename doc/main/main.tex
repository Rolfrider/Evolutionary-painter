\documentclass{article}
\usepackage[T1]{fontenc}
\usepackage[polish]{babel}
\usepackage[utf8]{inputenc}

\title{Ewolucyjny malarz }
\author{Rafał Kwiatkowski, Franciszek Sioma}


\begin{document}
\pagenumbering{gobble}
\maketitle

\section{Opis projektu}
\subsection{Cel projektu}
Celem projektu {\it Ewolucyjny Malarz} było stworzenie aplikacji wykorzystującej algorytm ewolucyjny. Aplikacja przyjmuje na wejściu kolorowy obrazek, a na wyjściu zwraca możliwe jak najdokładniejsze odwzorowanie obrazu wejściowego przy użyciu prostokątów RGBA. 

\subsection{Przyjęte założenia}
W algorytmie ewolucyjnym istotne jest to, aby zdefiniować czym będzie {\it Osobnik}. W naszym przypadku osobnikiem jest obraz stworzony z {\it n} prostokątów RGBA. Każdy prostokąt składa się z 8 wartości: 
\begin{description}
    \item[R] kanał czerwony
    \item[G] kanał zielony
    \item[B] kanał niebieski
    \item[A] kanał alpha (mówi o przezroczystości danego piksela)
    \item[X] pozycja względem osi poziomej
    \item[Y] pozycja względem osi pionowej
    \item[W] szerokość
    \item[H] wysokość       
\end{description}

Kolejnym niejasnym punktem może być sposób obliczania funkcji przystosowania ({\it J}). W naszym przypadku funkcja ta będzie przyjmować wartości od 0 do 1 i będzie obliczana na podstawie porównywania piksel po pikselu obrazu stworzonego przez osobnika z obrazem oryginalnym.
\subsection{Wkład autorów}
\begin{itemize}
    \item Obsługa obrazów - Rafał Kwiatkowski
    \item Algorytm Ewolucyjny - Franciszek Sioma
    \item Testy i eksperymenty - Rafał Kwiatkowski
    \item Dokumentacja - Franciszek Sioma
\end{itemize}
\subsection{Decyzje projektowe}
Algorytm zakłada dwa rodzaje krzyżowania osobników:
\begin{itemize}
    \item uśrednianie
    \item interpolacja
\end{itemize}
W naszym programie postanowiliśmy umieścić oba te sposoby, by móc przebadać ich wpływ na końcowy wynik algorytmu.
Taki sam zabieg dotyczy metod wybierania kolejnej generacji populacji. Zaimplementowane metody to:
\begin{itemize}
    \item metoda $\mu$ najlepszych
    \item metoda koła ruletki
    \item metoda selekcji rankingowej
\end{itemize}
Algorytm zwraca najlepszego osobnika wraz z jego wartością funkcji dopasowania. Program wyświetla obraz reprezentowany przez tego osobnika.
\subsection{Wykorzystane narzędzia i biblioteki}
Do napisania aplikacji użyliśmy języka Python w wersji: 3.8, dokumentacja została stworzona przy użyciu języka Latex, a IDE z którego korzystaliśmy to Visual Studio Code. Użyliśmy również systemu kontroli wersji Git.

Spis użytych bibliotek znajduje się w pliku {\it requirements.txt}.
Do najważniejszych z nich należy biblioteka Pillow wykorzystana do obsługi obrazów.
\section{Uruchamianie aplikacji i odtworzenie wyników testów}
W celu uruchomienia aplikacji należy wykonać komendę:
\begin{verbatim}
python app.py "simple_img.jpg" 30 20 45 1000 0.95 "inter" "best"
\end{verbatim}
Wówczas aplikacja zostanie wykonana dla wielkości polpulacji równej 30, w każdym z osobników będzie 20 prostokątów, podpopulacja wybierana z populacji do krzyżowania będzie składać się z 45 osobników, maksymalna liczba iteracji wyniesie 1000, a warunkiem stopu będzie dopasowanie na poziomie 95\%.
Algorytm użyje interpolacji przy krzyżowaniu, oraz użyje metody najlepszego dopasowania do wybrania kolejnej generacji populacji. Pozostałe dostępne wartości to "mean" - uśrednianie dla krzyżowania oraz "roulette" i "ranking" (odpowiednio metoda koła ruletki i selekcji rankingowej) dla wybierania.
\section{Eksperymenty}
Wykonaliśmy szereg testów, sprawdzających jak działa aplikacja w zależności od następujących parametrów:
\begin{itemize}
    \item liczebność populacji
    \item liczebność populacji $\mu$
    \item liczba prostokątów w osobniku
    \item strategia krzyżowania
    \item strategia wybierania kolejnej generacji populacji
\end{itemize}

\subsection{Eksperymenty i wyniki}
W celu uzyskania możliwie jak najbardziej prawdziwych wyników dla każdego przypadku testowego, każdy eksperyment z danymi parametrami był przeprowadzony 5 razy. Wszystkie testy zostały wykonane dla obrazu {\it simple\_img.jpg} oraz dla z warunkami zakończenia 1000 iteracji lub dokładność 99\%. Gdy nie było to przedmiotem eksperymentu to metodą krzyżowania było uśrednianie, a metodą wybierania - wybór najlepszych.

\subsection*{Wpływ liczebności populacji}
Test wpływu liczebności populacji został przeprowadzony dla populacji o wielkościach: 2, 10, 15, 30, 40, 50, 100, liczbie prostokątów w osobniku równej 20, wielkości populacji $\mu$ będącej 1,5 raza większej od wielkości testowanej populacji.

\subsection*{Wpływ liczebności populacji $\mu$}
Test wpływu liczebności populacji $\mu$ został przeprowadzony dla stałej wielkości populacji $\lambda$ równej 40 na podstawie której kolejne przypadki testowe był obliczane za pomocą wartości procentowych: 110\%(44), 130\%(52),
150\%(60), 200\%(80), 250\%(100), 300\%(120). Liczba prostokątów w osobniku była równa 20.

\subsection*{Wpływ liczby prostokątów w osobniku}
Następny test został przeprowadzony dla następujących liczb prostokątów w osobniku: 10, 20, 50, 100, 200, 300. Wielkość populacji wynosiła 10, podpopulacji 15. Pozostałe parametry zostały niezmienione względem poprzedniego testu. 

\subsection*{Wpływ użytej strategii krzyżowania}
Dla populacji $\lambda$ - 20, populacji $\mu$ - 30 i liczbie 20 prostokątów w osobniku testy zostały przeprowadzone dla uśredniania i interpolacji.

\begin{table}[]
    \centering
    \begin{tabular}{|c|c|c|}
    \hline
                 & Najlepszy osobnik & Średnio najlepszy osobnik \\ \hline
    Uśrednianie  & 89\%              & 83\%                      \\ \hline
    Interpolacja & 83\%              & 79\%                      \\ \hline
    \end{tabular}
    \caption{Wyniki testów krzyżowania}
    \label{tab:crossing}
\end{table}

\subsection*{Wpływ użytej strategii wybierania kolejnej populacji}
Dla populacji $\lambda$ - 20, populacji $\mu$ - 30 i liczbie 20 prostokątów w osobniku testy zostały przeprowadzone dla wyboru najlepszych, ruletki i metody rankingu.

\begin{table}[]
    \centering
    \begin{tabular}{|c|c|c|}
    \hline
                 & Najlepszy osobnik & Średnio najlepszy osobnik \\ \hline
    Uśrednianie  & 89\%              & 83\%                      \\ \hline
    Interpolacja & 83\%              & 79\%                      \\ \hline
    \end{tabular}
    \caption{Wyniki testów wybierania}
    \label{tab:picking}
\end{table}

\subsection{Wnioski}
\end{document}